\documentclass{article}
\usepackage{polski}
\usepackage[utf8]{inputenc}
\usepackage{graphicx}
\usepackage{amsmath}
\usepackage{multirow}
\usepackage{pgfplots}
\usepackage{longtable}
\usepackage{multicol}
\usepackage{booktabs, colortbl}
\usepackage{pgfplotstable}
\pgfplotsset{width=12cm, compat=1.9}
\newsavebox\ltmcbox
\newenvironment{fakelongtable}
        {\setbox\ltmcbox\vbox\bgroup
        \csname @twocolumnfalse\endcsname
        \csname col@number\endcsname\csname @ne\endcsname}
        {\unskip\unpenalty\unpenalty\egroup\unvbox\ltmcbox}

\title{Sprawozdanie}
\author{Julia Strużyńska 272981}
\date{December 2022}

\begin{document}

\maketitle
\section{Wprowadzenie Tematyczne}
Spadek swobodny to ruch ciała, który odbywa się pod wpływem siły grawitacji. Rozpoczyna się od spoczynku tj. z prędkością początkową równą zero. W pobliżu Ziemi dla ciał stosunkowo gęstych o aerodynamicznym kształcie spadek swobodny można traktować jako ruch jednostajnie przyspieszony z przyspieszeniem ziemskim g. Ruch można wtedy opisać za pomocą równania:

\begin{equation}
h(t)=h_0-\frac{(gt^2)} {2}
\end{equation}

gdzie:
\begin{itemize}
	\item \(h(t)\) – wysokość, na jakiej znajduje się ciało po czasie \(t\)
	\item \(h_0\) – wysokość z jakiej spada ciało
	\item \(t\) – czas spadania
	\item \(g\) – przyspieszenie ziemskie \(9,81 \frac{\text{m}} {\text{s}^2}\)
\end{itemize}

\maketitle
\section{Opis eksperymentu}

Tu znajduje się opis eksperymentu. Ładna ilustracja obrazująca spadek swobodny ciał na przykładzie monet (\ref{fig:spadek.swobodny}).

\begin{figure}[htbp]
    \centering
    \includegraphics[width=3cm]{monety.jpg}
    \caption{Spadające monety}
    \label{fig:spadek.swobodny}
\end{figure}

\maketitle
\section{Wyniki Pomiarów}
Tu znajdują się wyniki pomiarów. Tabela zawiera pomiary wysokości $h$ w czasie $t$.

\begin{multicols}{3}
\pgfplotstabletypeset[
begin table=\begin{fakelongtable}\begin{longtable},
end table=\end{longtable}\end{fakelongtable},
every even row/.style={before row={\rowcolor[gray]{0.9}}},
every head row/.style={before row=\toprule,after row=\midrule},
every head row/.append style={after row=\endhead},
every last row/.style={after row=\bottomrule},
]{csv.txt}

\end{multicols}

\newpage

\maketitle
\section{Wnioski}
Tu znajdują się wnioski.

\begin{tikzpicture}
\begin{axis}[ axis lines = left, 
              xlabel = {\(t [s]\)},
              ylabel = {\(h [m]\)},
              enlargelimits = false,]

\addplot [ domain=0:10, 
           samples=100, 
           color=red,]
         {x^2*9.81*0.5};
\addlegendentry{\(t^2*9.81*0.5\)}
\addplot+[only marks, 
          color=blue,
          mark=diamond,
          mark size=1.0pt]
table[meta=h]{csv.txt};
\end{axis}
\end{tikzpicture}
\end{document}

